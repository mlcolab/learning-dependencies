\documentclass{article}

% if you need to pass options to natbib, use, e.g.:
%     \PassOptionsToPackage{numbers, compress}{natbib}
% before loading neurips_2021

% ready for submission
\usepackage[preprint]{neurips_2021}

% to compile a preprint version, e.g., for submission to arXiv, add add the
% [preprint] option:
%     \usepackage[preprint]{neurips_2021}

% to compile a camera-ready version, add the [final] option, e.g.:
%     \usepackage[final]{neurips_2021}

% to avoid loading the natbib package, add option nonatbib:
% \usepackage[nonatbib]{neurips_2021}

\usepackage[utf8]{inputenc} % allow utf-8 input
\usepackage[T1]{fontenc}    % use 8-bit T1 fonts
\usepackage[colorlinks = true, 
		    linkcolor = blue,
		    urlcolor  = blue,
            citecolor = blue,
            anchorcolor = blue]{hyperref}       
% hyperlinks
\usepackage{url}            % simple URL typesetting
\usepackage{booktabs}       % professional-quality tables
\usepackage{amsfonts}       % blackboard math symbols
\usepackage{nicefrac}       % compact symbols for 1/2, etc.
\usepackage{microtype}      % microtypography
\usepackage{xcolor}         % colors
\usepackage{natbib}
\usepackage[pdftex]{graphicx}
\usepackage{siunitx} % Required for alignment
\sisetup{
  round-mode          = places, % Rounds numbers
  round-precision     = 2, % to 2 places
}
	
\bibliographystyle{unsrtnat}


\title{Can language models be used to predict educational concept dependencies?}

% The \author macro works with any number of authors. There are two commands
% used to separate the names and addresses of multiple authors: \And and \AND.
%
% Using \And between authors leaves it to LaTeX to determine where to break the
% lines. Using \AND forces a line break at that point. So, if LaTeX puts 3 of 4
% authors names on the first line, and the last on the second line, try using
% \AND instead of \And before the third author name.

\author{%
  Dominik Glandorf\\
  Matrikelnummer 6007407\\
  \texttt{dominik.glandorf@student.uni-tuebingen.de} \\
  \And
  Anastasiia Alekseeva\\
  Matrikelnummer 5994775\\
  \texttt{anastasiia.alekseeva@student.uni-tuebingen.de} \\
  \\
  GitHub repository: \url{https://github.com/dominikglandorf/learning-dependencies}
}

\begin{document}
\vspace*{-5mm}
\maketitle
\vspace*{-5mm}

\begin{abstract}
% What did we do?

% Why did we to it? What did we expect?

% How did we do it?

% What did we find out?

% What now?

\end{abstract}

\section{Introduction}
%% Why are we interested in learning dependencies?
% Problem of sequencing in Instructional Design:
Effective and efficient instruction does not only incorporate \textit{what} to teach but also \textit{how} to teach. Models of instructional design especially emphasize the importance of order of instruction. More precisely, prerequisites of educational content should be taught first \citep{morrison2019designing}.
% Definition KCs
As potential contents, Merill (1983) differentiated learning objectives such as facts, concepts, principles, rules, procedures, interpersonal skills and attitudes as well as their sole recall from their application. In this work we will focus on concepts and their relations. 
% Learning dependency
% Assumption: information about dependencies between concepts makes instructional sequencing more effective in terms of learning
If one concept is a prerequisite of another, we call this relation a learning dependency. For example, to understand the concept of a derivative knowledge about the concept of a function before will facilitate or even enable learning. When the dependencies are thought of as directed edges between nodes that represent concepts, a learning dependency graph emerges which is a special type of a knowledge graph \citep{wang2016using}. This graph is also called \textit{concept map} in the field of Learning Sciences.
% Mining of learning dependencies
Prerequisites can be inferred from learner behavior by testing their performance after being presented different instructional sequences (Pavlik et al., 2008, Vuong et al., 2011). However, this has the disadvantage of disengaging users with too difficult concepts before teaching easier or necessary ones. Experts usually dispose of the required knowledge about concepts to create concept maps. The high cost of expert knowledge motivates the automatic mining of learning dependencies from appropriate sources which is the core problem of this work.

% Advantage: Curriculum planning (Yang, 2015) especially for new topics, Automated Assessment (Wang 2015)

% What can we contribute?
% 1. Extract KC dependencies from textbooks
% 2. Bypass textbook mining by querying a pre-trained language model to disclose relationships between concepts

% What is our research question?
% How well can langauge models predict dependencies between key concepts compared to expert knowledge in textbooks?


\subsection{Related work}
% Field: Dependencies between concepts
% of interest: How are dependencies conceptualized?
\cite{talukdar2012crowdsourced} defined the prerequisite relation in terms of the consumption of information about concepts. Vuong (2011) if a better graduation rate given prerequisite knowledge is fulfilled. Concepts are often equated with Wikipedia articles (Talukdar and Cohen, 2012; Wang, 2015).
% Field: Textbook Knowledge Extraction
% of interest: How are relationships betweens concepts extracted?
% Field: Prompt Engineering
% of interest: How can we get internal representations out of a language model?

\section{Method}

\subsection{Information sources}
% Where did we get the data from?
% textbooks
For preprocessing the textbooks we used Wikifier \citep{brank2017annotating}.
% language models

\subsection{Structured information extraction}
% Data quality

% Visualizations

% Statistics

\section{Results}



\section{Discussion}
% What is our main result?

% What could be potential flaws?

% Q: Outlook and Future research ideas:% 


\bibliography{references.bib}

\end{document}
